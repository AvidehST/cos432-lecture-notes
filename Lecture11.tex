%!TEX root = InfoSec.tex
% Lecture 11: 15 October 2014
\sektion{11}{Firewalls and virtual private networks}

\subsektion{POODLE attack}
\begin{enumerate}
	\item Become a network MITM 
	\item inject code into an http page that the victim visits; code runs in victim's browser
	\item attacker's code connects repeated to (for example) https://mail.google.com/
	\item attacker forces use of legacy SSL3 protocol
	\item exploit SSL3 to see plaintext
	\item get victim's authentication cookie for gmail
	\item impersonate victim :(
\end{enumerate}

\textbf{Specifically, how is (4) done?}\\
Version negotiation: both endpoints discuss which version of the secure protocol to use. Negotiation must begin in cleartext since you can't start encrypting until you know which encryption methods the other party can use. MITM can pretend that both sides can only use SSL3, called a \textbf{version downgrade attack}.

Defense: sign the exchange and ensure that what A sent is what B saw.

\textbf{The downgrade dance}\\
If, for some reason, negotiation fails a number of times, both sides just use SSL3. A MITM attack can also exploit this to force both sides to use SSL3.

\textbf{Okay, so now how is (5) done?}\\
How does SSL3 perform ``authenticated encryption''?\\
Block cipher in CBC mode: $E(x || M(x))$. An adversary can mess around with the padding and not get caught because the padding is garbage (as opposed to 10*). There's a 1/256 chance of finding one byte. The adversary can then shift the stream of bytes and read the message.

\subsektion{Firewalls and VPNs}
Perimeter defense
\begin{itemize}
	\item separate outside from inside
	\item monitor the boundary
	\item block incoming stuff that's bad or ``weird''
\end{itemize}

Firewall = perimeter defense for network

Simple firewall: inside = ``clients'' only (end users; eg. tablets, phones, computers)\\
Policy: allow connection if initiated from the inside.\\
Implementation: packet filter that allows outgoing packets and blocks incoming if non TCP or TCP SYN (connection initiation)

\subsektion{Firewall Admin/Security}
This is important because it controls the network communication, and is a MITM by definition. It (1) shuts off/limits network services on the firewall itself and (2) should accept connections only from the inside.

Initiating a TCP connection involves a 3 step package exchange handshake. 

\textbf{Network Address Translation}\\
aka NAT\\
Your whole network shares a single ``real'' IP address. Outsiders cannot target a machine inside the network because they aren't even addressable from the outside. It's also a nice setup because you only need to buy one IP address from the network provider.

NAT keeps a translation table of inside IP addresses to their outside IP address equivalents. For example, $10.0.0.13:85 \Rightarrow 11.12.13.14:2015$ where the former is the inside address and the latter is the outside.

NAT maintains the illusion of IP addresses for inside destinations so that both sides can communicate without know NAT is in the middle. 

\sidenote {
	\textbf{Building/maintaining translation tables}\\
	Add an entry when an outgoing packet is initiating a TCP connection.\\
	Tear down an entry when
	\begin{itemize}
		\item see an orderly shutdown of a TCP connection
		\item unused for a long time\\

		Keep-alives (little packets sent periodically) exist to keep this from happening\\

		OR if endpoints need to use the connection again, the connection will be broken. This is fine.
	\end{itemize}
}

\sidenote {
	\textbf{Fictitious IP address ranges}\\
	Four numbers, each represent a byte.\\
	10.*.*.*\\
	192.168.*.*
}

\subsektion{Supporting your own servers}
If you want to have your own server inside the autonomous system, how might you set it up?
\begin{enumerate}	
	\item Exception in your NAT rules to allow outside to access your server.

		Downside: If your server is compromised, the attacker has access to your internal network/have gotten past your firewall
	\item Put the server on the outside

		Downside: You don't have direct access to the server, also probably more expensive. Though, we're trending towards this approach
	\item DMZ approach: inside network -- firewall -- server -- firewall -- THE INTERNET

		Downside: It's more complicated since it has more firewalls
	\item Single firewall, but with two servers: public and private. Incoming emails would be delivered to the public server. A siphon pulls messages from the public server into the private server, where users can get to the emails. The siphon is presumably strong and encrypted.
\end{enumerate}