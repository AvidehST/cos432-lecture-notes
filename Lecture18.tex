%!TEX root = InfoSec.tex
% Lecture 18: 17 November 2014
\sektion{18}{Malware}

\sidenote {

	\textbf{Stuxnet} was a piece of malware that reportedly infected 1/5 of Iran's nuclear reactors.\\
}
% Stuxnet -- how it worked (see HN article lol)

\textbf{Malware Taxonomy}
\begin{itemize}
	\item Doesn't spread + requires host: Trojan, Rootlet
	\item Doesn't spread + runs independently: Keylogger, spyware
	\item Spreads + requires host: virus
	\item Spreads + runs independently: worm
\end{itemize}

\textit{What are the goals of malware?}\\
Money: spam, steal data and credentials\\
Launch attacks - DoS, cyberwar

\textbf{Why does Window suffer the most?}
\begin{itemize}
	\item larger market share
	\item more bugs, greater attack surface
	\item usability and backward compatibility emphasized over security
	\item fewer versions, more homogenous "monoculture, like in agriculture"
\end{itemize}

\subsektion{Hosts}
\begin{itemize}
	\item executable files
	\item boot sector
	\item macros (like in a word or excel doc)
\end{itemize}

Anything computationally powerful to allow self-replication
(think Facebook statuses too!) 

\begin{definition}
Payload: Code that attacks
\end{definition}

\begin{definition}
Infection mechanism: how it spreads
\end{definition}

\subsektion{Infection analysis}
People are either susceptible, infected, or recovered (not susceptible).

\textbf{Viruses life cycle}
\begin{enumerate}
	\item Dormant -- can then reproduce, or can attack once it's triggered
	\item Reproduction -- then infects others
	\item Infection -- can then go dormant
\end{enumerate}

\textbf{Worms}
\begin{enumerate}
	\item Target locator; find vulnerable machines

	Example: For emails, scan email address books/buy lists of email addresses. For IP addresses; scan IPs
	\item Infection propagator; compromise victim and transfer copy

	Example: an email with attachments
	\item Lifecycle manager; command and control the worm
\end{enumerate}

\sidenote {
	\textbf{SQL slammer}\\

	This was a worm that was only 404 bytes!!!! Its doubling time was less than 10 seconds and randomly scanned for IP addresses to locate targets. 90\% of susceptible machines on the internet were infected in 10 minutes

	It caused a buffer overflow in SQL servers.\\
}

\textbf{Flash worm}
Pre-scans the entire internet to pre-compute the infection tree. Has a branching factor of 10 and a height of about 7 (depends on number of vulnerable hosts).

Each infection knows its subtree address so that 1 million hosts might be infected in less than 2 sec

\textbf{Rootkit}
Tools used by attackers after they compromise a system.\\
Purpose: 
\begin{itemize}
	\item hide presence of attacker
	\item allow for return of attacker at a later date
	\item gather info about environment
\end{itemize}
For example, a kernal rootkit might list processes and modules and intercept API calls from applications.

\subsektion{Defenses}
\textbf{Antivirus}
\begin{itemize}
	\item \textbf{Traditional: signature based}

	Uses substring/regex match to check software against database of byte-level or instruction-level signatures, one for each malware or family.

	They are speedy, and often manually compiled or updated. They can't be proactive; are looking only for known attacks

	\textbf{Evading this}
	\begin{itemize}
		\item encryption -- encrypt malware body (so it looks random), and decrypts upon execution
		\item polymorphism -- decryption routine also looks different each time
		\item metamorphism -- different instructions, same semantics (eg. \texttt{SUB eax, eax == XOR eax, eax})
	\end{itemize}
	
	\item \textbf{Sandboxed emulation}

	Runs code in a sandbox and checks for malware signatures in memory. This defeats polymorphism, but is slow.

	\item \textbf{Behavioral Analysis}

	Detects if a piece of code tries to do suspicious things, such as modifying the registry, attempting to edit system files, attemptint to hid or replicate, or connecting to known malware IPs/hosts.
\end{itemize}

\textbf{Network based defenses} \\ 
Firewalls and etc. 

\textbf{Host-based defenses}\\
eg. nicer UI so user isn't easily tricked



