% Lecture 8: 10 October 2012
% System and software security - done with Crypto
\sektion{8}{Access control}
How you reason about and enforce rules about who's allowed to do what in the
system

\begin{definition}{Trusted subsystem}\\
A program, with state, that is \underline{isolated} from
the rest of the world, and interacts via \underline{declared interfaces}
\end{definition}

Access control: SUBJECT wants to do VERB on OBECT. Okay?

This deals with authentication (Who is asking?), not authorization (Does that
person have permission?)

SUBJECT: a running program\\
VERB: an operation/API call\\
OBJECT: resource on the system

Policy: a set of (S,V,O) triples that are allowed
\begin{itemize}
    \item How to determine policy? (\underline{should})
    \item How to enforce policy? (\underline{is})
\end{itemize}
One data structure: Access Control Matrix\\
\parbox[c]{5cm}{\makebox[5cm]{$\longleftarrow$ objects $\longrightarrow$}\\
\parbox[c]{1cm}{$\uparrow$\\
\begin{sideways}subjects\end{sideways}\\
$\downarrow$}
\makebox[.5cm]{}
\fbox{$V_1, V_2$}}

\subsektion{Subjects and labels}
\begin{itemize}
    \item subject = running program
    \item often, give labels to subjects and set policy based on labels

    (+) reduces matrix size\\
    (+) easier to make policy\\
    (--) oversimplifies? Suppose: label = userid and means program is running
    ``for'' userid. Alice runs a program written by Bob (example: Alice uses a
        text editor written by Bob to edit Alice's secret file). What label?
    \begin{itemize}
        \item If treat as Alice: Bob's code can send Alice's secret data to Bob
        \item If treat as Bob: Alice can't edit her secret file
        \item If treat as Bob but special for this file: none of the labelling
        benefits
        \item If treat as intersection of privileges: get all the drawbacks
    \end{itemize}
\end{itemize}
Store access control info:
\begin{itemize}
    \item as AC matrix - note that this will be very sparse
    \item as ``profiles'' - for each user, list of what subject can do
    \item as Access Control List (ACL) - for each object, list of (Verb,
            Subject) pairs (who can do what to it). This is typically used
    because small and simple in practice.
\end{itemize}
Who sets policy?
\begin{itemize}
    \item centralized (``mandatory'') - done by an authority

    (+) done by a well-trained person\\
    (+) might be required (ethical, legal, or contractual obligations)\\
    (--) inflexible
    \item decentralized (``discretionary'') - each object has an
    \underline{owner}, owner set ACLs

    (+) flexible\\
    (--) every user makes security decisions
\end{itemize}
Groups and Roles:\\
Group is a set of people with some logical basis; role is group with one
member\\
Advantages:
\begin{itemize}
    \item makes ACL smaller
    \item change in status naturally causes change in access to resources
    \item ACL explains why you have access
\end{itemize}
Roles can be hidden temporarily, ``wearing different hats'' (useful for testing)

\subsektion{Capabilities}
A different approach to access control: controls access without identification,
like a physical key, ``the bearer has permission to read foo''

Sometimes make them revokable, but that's a pain to do in practice

Implementation: crpyto
\begin{enumerate}
    \item system has a secret key $k$, capability = MAC($k$, verb || object)
    \item public-key: one party grant permission (makes digital signature),
    another party control access (makes sure handed valid capability - verifies
    signature)
\end{enumerate}
Implementation: OS table\\
OS stores a list of your capabilities; Alice makes a system call to give Bob
capabilities for a certain file (file descriptors used to say you've an open
file are an example)

Implementation: in a type-safe programming language (like Java), pointer to an
object is a capability
\subsektion{Logic-based authorization}
Define a formal logic, with primitives for
\begin{itemize}
    \item principals/users
    \item objects
    \item delegation
    \item time
\end{itemize}
To get access, submit a proof that you are authorized

Parties make statements by digital signing

System allows for great complexity in policies, but only need simple proof-
checking mechanism to make it work. But also need to work out a way to get
people able to write these statements, and deal with possible large proof size

Caveat: people don't actually use complicated access control mechanisms, and
usually just leave them as the defaults or make it visible to the whole world

Want to come up with a system which infers what the user wants from the way the
user behaves (best if not visible to user)
