% Lecture 1: 17 September 2012
\sektion{0}{Course Information}
Professor: Ed Felten

Website: \url{http://bit.ly/Ribzeh} and Piazza

Lectures: 11am Monday, Wednesday in Computer Science 104

\sektion{1}{Message Integrity}
\subsektion{Sending messages}
\ovalbox{Alice} $\xrightarrow{m}$
    \ovalbox{Mallory} $\xrightarrow{?}$
    \ovalbox{Bob}

Assume: ({\bf threat model}, what adversary can do and accomplish; always assume
         that Mallory is malicious in the most devious possible way, as opposed
         to random error)
\begin{itemize}
    \item Mallory can see and forge messages
    \item Mallory wants to get Bob to accept a message that Alice didn't send
\end{itemize}

\sidenote{
    {\bf CIA Properties}
    \begin{itemize}
        \item Confidentiality: trying to keep information secret from someone
        \item Integrity: making sure information hasn't been tampered with
        \item Availability: making sure system is there and running when needed
    \end{itemize}
}
In this problem, the goal is integrity (we can't get the other two)

\sidenote{
    {\bf Role of stories in security:}
    \begin{itemize}
        \item Pro: easy to follow
        \item Cons:
        \begin{itemize}
            \item forget ``Alice'' is a computer (no common sense)
            \item forget ``Alice'' is a person + computer (one may have some
                    knowledge that other doesn't, e.g. knowledge divergence in
                    phishing attack
            \item might root for one side or the other and lose impartiality
        \end{itemize}
    \end{itemize}
}

What to send:\\

\ovalbox{Alice} $\xrightarrow{(m, f(m))}$
    \ovalbox{Mallory} $\xrightarrow{(a,b)}$
    \ovalbox{Bob} : accept $a$ iff $f(a) = b$

$f$ is a {\bf Message Authentication Code (MAC)}\\

Properties $f$ needs to be a secure MAC:
\begin{enumerate}
    \item deterministic (Bob needs to get the same answer that Alice got)
    \item easily computable by Alice and Bob
    \item not computable by Mallory (else Mallory can send $(x, f(x))$ for any
        $x$ he wants)
\end{enumerate}

Choosing $f$:
\begin{itemize}
    \item picking something that you think Mallory won't guess is risky since
        you can't draw any confident conclusions
    \item use a random function:
        \begin{table}[!h]\centering\begin{tabular}{r|ll}
            input & output &\\
            \cline{1-2}
            $\emptyset$ & 01011... & $\leftarrow$ 256 coin flips\\
            0 & 101... &\\
            1 & ... &\\
        \end{tabular}\end{table}

\sidenote{
    {\bf ``secure MAC game'': us vs. Mallory}

    \hspace*{0.5 cm} repeat until Mallory says ``stop'': \{\\
        \hspace*{1 cm} Mallory chooses $x_i$\\
        \hspace*{1 cm} we announce $f(x_i)$\\
    \hspace*{0.5 cm} \}\\
    \hspace*{0.5 cm} Mallory chooses $y \not\in \{x_i\}$\\
    \hspace*{0.5 cm} Mallory guesses $f(y)$: wins if right

    Say $f$ is a secure MAC if Mallory can't do better than random guessing
    \bigskip
    \begin{theorem*}{A random function is a secure MAC.}\end{theorem*}
    \emph{Intuition:} Mallory asks to reveal certain entries, but for $y$
        Mallory is trying to guess the result of the coin flips
}
    \item use a pseudorandom function:

        {\bf pseudorandom function (PRF)}: ``looks random'', ``as good as
            random'', practical to implement

        typical approach:
        \begin{itemize}
            \item \underline{public} family of function $f_0, f_1, f_2, \dots$
            \item \underline{secret} key $k$
            \item \underline{use} $f_k$
        \end{itemize}

\sidenote{
    {\bf Kerckhoffs's principle:}

    Use a public function family and a randomly chosen secret key.
    \bigskip
    Advantages:
    \begin{enumerate}
        \item can quantify probability that key will be guessed
        \item different people can use the same functions with different keys
        \item can change key if needed (something goes wrong)
    \end{enumerate}
}
\sidenote{
    {\bf ``PRF game'' against Mallory}:

    \hspace*{0.5 cm} we flip a coin secretly to get $b \in \{0,1\}$\\
    \hspace*{0.5 cm} if $b = 0$, let $g = $ random function\\
    \hspace*{0.5 cm} else, $g=f_k$ for random $k$\\
    \hspace*{0.5 cm} repeat until Mallory says ``stop'': \{\\
        \hspace*{1 cm} Mallory chooses $x_i$\\
        \hspace*{1 cm} we announce $g(x_i)$\\
    \hspace*{0.5 cm} \}\\
    \hspace*{0.5 cm} Mallory guesses $b$: wins if right

    $f$ is a PRF if Mallory can't do better than 50/50 (actually, allow a tiny
            advantage)\\

    Note: Mallory can always win by exhaustive search of the $f_k$'s, so need to
    limit Mallory to ``practical''\\

    \begin{theorem*}{If $f$ is a PRF, then $f$ is a secure MAC}\end{theorem*}
    \begin{proof} By contradiction. \end{proof}
}
\end{itemize}

What to send (new):\\

\ovalbox{Alice} $\xrightarrow{(m, f_k(m)}$
    \ovalbox{Mallory} $\xrightarrow{(a,b)}$
    \ovalbox{Bob} : accept $a$ iff $f_k(a) = b$

Assumptions:
\begin{enumerate}
    \item $k$ is kept secret from Mallory
    \item Alice and Bob both know $k$
\end{enumerate}

\subsektion{Do PRF's exist?}
Answer: maybe/ we hope so (some functions haven't lost yet)\\

Here's one: HMAC-SHA256
$$f_k(x) = S((k \xor z_1) || S((k \xor z_2) || x))$$
where $z_1 = 0x3636\dots$, $z_2 = 0x5c5c\dots$ (note that $||$ is concatenation)
and $S$ is ``SHA-256'': start with ``compression function'' $C$, taking 256 and
512 bits in, outputting 256 bits

\makebox[2cm]{}\framebox[8cm]{input}\framebox[2cm]{pad}\\
\makebox[2cm]{}\framebox[2cm]{}\framebox[2cm]{}\framebox[2cm]{}\framebox[2cm]{}
\mbox{512 bit blocks}\\
\makebox[2cm]{}\makebox[2cm]{$\Downarrow$}\makebox[2cm]{$\Downarrow$}
\makebox[2cm]{$\Downarrow$}\makebox[2cm]{$\Downarrow$}\\
\makebox[2cm]{const $\rightarrow$}\makebox[2cm]{C}\makebox[.2cm]{$\rightarrow$}
\makebox[1.4cm]{C}\makebox[4.4cm]{$\cdots$}\makebox[2cm]{output}

Note: This is subject to length extension attacks\\

``cryptographic hash'' (MD5, SHA-1, SHA-?) are dangerous to use directly because
they don't have the properties you think/want then to have.\\

Better: use a PRF even if $k$ is non-secret

\subsektion{Multiple Alice - Bob messages}
How to deal with Mallory sending messages out of order or resending old messages
\begin{enumerate}
\item append sequence number to each message:\\
        Alice sends $m_0' = (0, m_0)$, $m_1' = (1, m_1)$
\item switch keys per message
\end{enumerate}
