%!TEX root = InfoSec.tex
% Lecture 7: 29 September 2014
\sektion{6}{Authenticating people}
\sidenote{
    {\bf SHA-3}

NIST: 1997 new standardization effort to pick SHA-3

recently keccak picked
\begin{itemize}
    \item fast to implement to implement in software, and really fast in
        hardware
    \item in practice, probably will be implemented in software, but brute force
        search to break it will probably be done in hardware -- slight
        conspiracy theory that NIST picked so as to advantage attackers with
        larger resources
\end{itemize}
}
Authenticating: Make sure someone is who they claim to be

Three basic approaches, relying on
\begin{enumerate}
    \item something you \underline{know} (mother's maiden name)
    \item something you \underline{have} (prox)
    \item something you \underline{are} (``biometrics'')
\end{enumerate}
\subsektion{Something you know: passwords}
Password threat model:
\begin{itemize}
    \item user picks a password and remembers it
    \item to log in, user gives name, password
    \item adversary wants to log in as user
    \item adversary \underline{might} be able to compromise server
\end{itemize}
First approach:
\begin{itemize}
    \item server has ``password database" of (name, password) pairs
    \item system verifies match
    \item Drawback: if adversary sees database, he/she can impersonate any user
\end{itemize}
Attacks (to get a user's password):
\begin{itemize}
    \item guessing attack\\
        online: try to log in as user\\
        offline: get password DB, computational search over passwords
    \item trick the user, or someone else who knows the password, into telling you -- surprisingly effective (``social
            engineering'')
    \item impersonate server, get user to ``log in'' to you (``spoofing'',
            ``phishing'')
    \item online guessing: try to log in with guessed name and password
    \item offline guessing: get password DB, computational search over passwords
    \item if user wrote down password, read it
    \item change the password database (somehow)
    \item watch the user log in, see what user types (``shoulder surfing'')
    \item compromise the user's device (somehow) and record actions (e.g. ``key logging")
    \item get user's password from one site, and try it on another
    	\begin{itemize}
		\item most users have between 3-5 passwords they reuse
	\end{itemize}
\end{itemize}
Countermeasures:
\begin{itemize}
    \item teach users not to divulge passwords (such as having a box saying
            ``AOL will never ask you for your password")
    \item make guessing harder
    \begin{itemize}
    	\item implement a time delay after password failure (e.g. 2 seconds); this will slow down guessing attacks
        \item limit number of failed attempts (``velocity control''), only for
                online guessing
        \item avoid informative error message if user fails to log in (so don't
                say username was right but password wrong)
        \item vs. offline guessing: slow down the verification code
        	\begin{itemize}
		\item compute $Hash^n$(PRF(...)) to verify password
		\item might slow verification by a factor of 1000
	\end{itemize}
            % tricky example: password doesn't match or memory error if does 
            % character-by-character comparison and put your password right
            % before unaccessible memory location so it'll memory error if first
            % character right; can then shift to get next part of prefix
    \end{itemize}
    \item server stores hash(password) rather than password, so password
            database doesn't convey passwords
    \item often, iterate hash: H(H(H(H(H...(password)...)))); slows brute-force
            search, but adversary can try ``dictionary attack'' -- hash many
            common passwords and build a handy retrieval data structure
    \item to frustrate dictionary attacks, use a ``salt'': for each user,
            generate a random value $S_u$, then store in password database
            (name, $S_u$, Hash(password, $S_u$ || password))

            Then an attacker would need to build a dictionary for each user

            Note that salt is in password database and it is convenient to keep
            secret, but hopefully password is strong enough for this to be okay
            even if the salt is leaked
            
            Note: in this model, the server doesn't store password
\end{itemize}
Guessing is a serious problem in practice: people pick lousy passwords, and
attackers get more powerful all the time by Moore's law
% Prof Felten's story
% coffee and begging day's password for a week if sysadmin guessed your password

% Some study on univeristy passwords gave for one univeristy beer, love, hockey,
% jesus top passwords

Reducing guessability:
\begin{itemize}
	\item hard to quantify guessability
	\item only sure way: make password random, chosen from a large space (these are usually hard to remember)
	\item format rules (e.g. special character and at least one uppercase character)
	\item require password to be longer (probably better than format rules)
\end{itemize}

Password hygiene
\begin{itemize}
    \item like key hygiene
    \item change periodically and avoid patterns (``password1'', ``password2'', ...)
    \item expire idle sessions (walk-away problem)
    \item require old password to change password
\end{itemize}

What if user forgets password?
\begin{itemize}
    \item if hashed password is stored, can only set a new one
    \item else, can tell them password, BUT how do you know it's not an
            impostor?
    \item clever solution by Gmail: if all else fails, we'll give you a new
            password, but you'll need to wait before trying to log in again.
            Then legitimate user may log in and see a warning during that time
\end{itemize}

Preventing spoofing:
\begin{itemize}
     \item multi-factor authentication: password + something else (e.g. token, app) 
     \item Evidence-based (Bayesian) authentication: treat password entry as evidence, but not 100\% certainty
     	\begin{itemize}
	\item then use as much other evidence as possible (e.g. geolocation)
	\item other examples: device identity, software version, behavior patterns (especially atypical behavior)
	\item if confidence is too low, get more evidence
	\end{itemize}
    \item distinctive per-user display
    \item distinctive unspoofable action before login

        Windows CTRL-ALT-DEL before every time you enter password, always taking
        you to legitimate login screen
    \item challenge-response protocol: (sign a challenge value)
    	\begin{itemize}
	\item advantage: eavesdropper can't replay a log-in session
	\item spoofer can't impersonate the user later
	\end{itemize}
    \item use one-time-passwords

        security advantage, but logistical disadvantages: server has to serve
        more stuff, might run out at an inconvenient time
    \item hash-chain: user generates random value $x_0$ then chain with
        $x_{i+1} = H(x_i)$. The one-time passwords are
        $x_{n-1}, x_{n-2}, \dots, x_0$ in this order, and the server checks that
        each password is the hash of the next password. User remembers $x_0$
        and where in chain they are.
        \begin{table}\centering\begin{tabular}{ccc}
            user & & system\\
            \hline
            password (p) & & random challenge (r) \\
            & $\xrightarrow{\text{name}}$ & \\
            & $\xleftarrow{r \text{ (random)}}$ & \\
            & $\xrightarrow{\text{PRF} (p,r)}$ &
        \end{tabular}\end{table}
    \item password + Diffie-Hellman: SPEKE (Simple Password Exponential Key
            Exchange)

        Use D-H with public prime $p$, server stores
        $g = (\text{Hash(password)})^2 \mod p$

        Results:
        \begin{itemize}
            \item user, server get shared secret from D-H
            \item MITH attack doesn't work
            \item user only has to remember a password, not a key
        \end{itemize}
\end{itemize}
\subsektion{Something you have}
Typically, tamper-resistant device stores a key or some cryptographic secret.

It does crypto to prove the user has it.
\subsektion{Something you are}
\begin{definition}{biometric}

Measuring aspect of user's body: fingerprint, iris scan, retina scan, finger
length, voice properties, facial features, hand geometry, typing patterns, gait
\end{definition}
Basic scheme:
\begin{itemize}
    \item enroll user: take a few measurements, compute ``exemplar''
    \item later, when user presents self, measure, compare to exemplar; compute
            ``distance'' to exemplar
    \item if ``close enough'', accept as valid user, else reject: tradeoff in
    threshold between false accepts and false rejects
\end{itemize}
Drawbacks:
\begin{itemize}
    \item hard/impossible to follow good key hygiene (can't change aspects of user's body)
    \item often requires physical presence
    \item spoofing attacks; make image of body part, faking tempertature,
            inductance, etc. (melted gummy bears moulded into finger-shape...)
    \item measurement is only approximate (need to control false positives and false negatives)
    \item publicly obversable (eg. DNA, fingerprints)
\end{itemize}
