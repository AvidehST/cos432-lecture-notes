% Lecture 4: 26 September 2012
\sektion{4}{Asymmetric key cryptography}
Symmetric key: use the same key on both ends

Problems:
\begin{itemize}
    \item Integrity: Alice sending to Bob, Charlie, Diana, ...

        If Alice, Bob, Charlie, Diana all have key $k$, then Bob could compute
        a MAC on a message and deliever a message to Charlie and Diana, thereby
        forging a message

        It would be nice if Alice were the only one who could send a verified
        message without needing to append everyone's integrity key (one per
        recipient)
    \item Confidentiality: maybe only Alice should be able to decrypt a message
\end{itemize}

Asymetric scheme: 1976, Diffie-Hellman(-Cox for British military)
\begin{itemize}
\item One key for encrypting, another for decrypting
\item One key for MAC, another for verifying it
\end{itemize}

\begin{definition}
{\bf ``public-key'' cryptography}

    Almost always:
    \begin{itemize}
        \item Generate key-pair such that can't derive one key from the other
        \item One key is kept private (only Alice knows it)
        \item Other key is public (everyone knows it)
    \end{itemize}
\end{definition}

\subsektion{RSA algorithm}
\begin{itemize}
    \item Best-known, most used public key algorithm
    \item 1978, Rivest-Shamir-Adleman
\end{itemize}

{\bf How it works:}

To generate an RSA key pair,
\begin{enumerate}
    \item Pick large secret primes $p$, $q$ (randomly chosen, typically 2048
            bits)

        Done by generating odd numbers in range and testing if prime, throwing
        away if not prime and trying again. Primes are dense enough that this
        isn't too bad, and primality testing is also okay on time.
    \item Define $N = pq$

        Useful fact: if $p$, $q$ are prime, for all $0 < x < pq$,
        $$x^{(p-1)(q-1)} \mod{pq} = 1$$
    \item Pick $e$ such that $0 < e < pq$, $e$ relatively prime to $(p-1)(q-1)$
    \item Find $d$ such that $ed \mod{(p-1)(q-1)} = 1$
\end{enumerate}
The public key is $(e, N)$ and the private key is $(d, N) [+(p,q)]$.

To encrypt or decrypt with public key:
$$\text{RSA}((e,N), x) = x^e \mod N$$
To encrypt or decrypt with private key:
$$\text{RSA}((d,N), x) = x^d \mod N$$

\begin{theorem*}``It works''
\end{theorem*}
\begin{proof}
    \begin{align*}
    \text{RSA}&((e,N), \text{RSA}((d,N), x))\\
    &= (x^d \mod{pq})^e \mod{pq}\\
    &= x^{de} \mod{pq}\\
    &= x^{a(p-1)(q-1)+1} \mod{pq} \text{, for some $a$}\\
    &= (x^{(p-1)(q-1)})^a x \mod{pq}\\
    &= (x^{(p-1)(q-1)} \mod{pq})^a x \mod{pq}\\
    &= 1^a x \mod{pq}\\
    &= x \mod{pq}\\
    &= x \text{, given $0 < x < pq$}
    \end{align*}
\end{proof}
Best known attack is to try factoring $N$ to get $p$, $q$
\subsektion{Why not use public-key always?}
\begin{itemize}
    \item It's slow ($\sim$1000x slower than symmetric)
    \item Key is big ($\sim$4k bits)
\end{itemize}
\subsektion{How to use public-key crpyto}
For confidentiality: (``your eyes only'')
\begin{itemize}
    \item Encrypt with public key
    \item Decrypt with private key
\end{itemize}
For integrity: (``digital signature'')
\begin{itemize}
    \item ``Sign'' by encrypting with private key
    \item ``Verify'' by decrypting with public key
\end{itemize}
\subsektion{Secure RSA}
!! Warning: Not secure as described above, need to fix !!\\

Problem 1:

Suppose $(e, N) = (3, N)$. Given ciphertext $8$ that was encrypted with $(3, N)$
it's trivial that $x^3 \mod N =8$ has $x = 2$. This shows that you may run into
trouble when encrypting small messages.\\

Problem 2 (Malleability):
\begin{align*}
\text{RSA}&((d,N),x) \cdot \text{RSA}((d,N),y) \mod N\\
&= (x^d \mod N)(y^d \mod N) \mod N\\
&= (xy)^d \mod N\\
&= \text{RSA}((d,N), xy)
\end{align*}

Adversary could use this to win the game defining security of the cipher

\begin{definition}
{\bf Malleability}

Adversary can manipulate ciphertext, get predictable result for decrypted
plaintext.

This is usually bad, but sometimes we want a malleable cipher (for some
application)
\end{definition}

Lesser problems:
\begin{itemize}
    \item Same plaintext results in same ciphertext (deterministic)
    \item No built-in integrity check
\end{itemize}

To solve all these problems, add a preprocessing step before encryption. The
standard way is call OAEP (Optimal Asymmetric Encyption Padding):
\begin{enumerate}
    \item Generate 128 bit random value, run through PRG $G$
    \item XOR with message padded with 128 bits of zeros
    \item Run result through PRF $H$, a hash function with announced key
    \item XOR with the random bits
    \item Concatenate result and send to RSA encyption
\end{enumerate}

\makebox[2cm]{}\makebox[1.5cm]{128 bits}\makebox[.5cm]{}\makebox[2cm]{128 bits}\\
\framebox[3.5cm]{\makebox[2cm]{message}\textbar\makebox[1.5cm]{000...}}\makebox[.5cm]{}\framebox[2cm]{random}\\
\makebox[6cm]{\makebox[3.5cm]{$\downarrow$}\makebox[.5cm]{}\makebox[2cm]{$\downarrow$}}\\
\makebox[1.5cm]{}\makebox[.5cm]{$\oplus$}\makebox[2.8cm]{$\longleftarrow\quad G\quad\longleftarrow$}\makebox[.4cm]{$\downarrow$}\\
\makebox[6cm]{\makebox[3.5cm]{$\downarrow$}\makebox[.5cm]{}\makebox[2cm]{$\downarrow$}}\\
\makebox[1.5cm]{}\makebox[.5cm]{$\downarrow$}\makebox[2.8cm]{$\longrightarrow\quad H\quad\longrightarrow$}\makebox[.4cm]{$\oplus$}\\
\makebox[1.5cm]{}\makebox[.5cm]{$\downarrow$}\makebox[2.8cm]{\rule[-0.1cm]{2.8cm}{0.01cm}}\makebox[.4cm]{$\downarrow$}\\
\makebox[1.5cm]{}\makebox[.5cm]{}\makebox[2.8cm]{$\downarrow$}\makebox[.4cm]{}\\
\makebox[1.5cm]{}\makebox[.5cm]{}\makebox[2.8cm]{to RSA encrpytion}\makebox[.4cm]{}\\

Also add the reverse as a postprocessing step after decryption:

\makebox[1.5cm]{}\makebox[.5cm]{}\makebox[2.8cm]{from RSA encrpytion}\makebox[.4cm]{}\\
\makebox[1.5cm]{}\makebox[.5cm]{}\makebox[2.8cm]{$\downarrow$}\makebox[.4cm]{}\\
\makebox[1.5cm]{}\makebox[.5cm]{$\downarrow$}\makebox[2.8cm]{\rule[0.3cm]{2.8cm}{0.01cm}}\makebox[.4cm]{$\downarrow$}\\
\makebox[1.5cm]{}\makebox[.5cm]{$\downarrow$}\makebox[2.8cm]{$\longrightarrow\quad H\quad\longrightarrow$}\makebox[.4cm]{$\oplus$}\\
\makebox[6cm]{\makebox[3.5cm]{$\downarrow$}\makebox[.5cm]{}\makebox[2cm]{$\downarrow$}}\\
\makebox[1.5cm]{}\makebox[.5cm]{$\oplus$}\makebox[2.8cm]{$\longleftarrow\quad G\quad\longleftarrow$}\makebox[.4cm]{$\downarrow$}\\
\makebox[6cm]{\makebox[3.5cm]{$\downarrow$}\makebox[.5cm]{}\makebox[2cm]{$\downarrow$}}\\
\framebox[3.5cm]{\makebox[2cm]{$m'$}\textbar\makebox[1.5cm]{$z'$}}\makebox[.5cm]{}\framebox[2cm]{$r'$}\\

Reject if $z'$ is not all zero, otherwise throw away $r'$ and let $m'$ be the
result of the decyption. $m'$ should at this point be equal to the original
message.

Other things to clean up:
\begin{itemize}
    \item Key size
    \begin{itemize}
        \item To get a big enough key space, need lots of possible primes
        \item Factoring is better than brute force
        \item Factoring algorithms might get better, so build in cushion in key
            size to account for incremental improvements in these algorithms.
        \item Today, 2048-bit primes seem okay
    \end{itemize}
    \item Useful performance trick
    \begin{itemize}
        \item $e = 3$ and make sure $p$ and $q$ are chosen
    such that 3 is relatively prime to $p-1$ and $q-1$
        \item This is extra-big win
    for digital signatures since verify is the common case.
        \item But: what if OAEP disappears from your code?

        Use $e=65537=2^{16} + 1$ instead
    \end{itemize}
    \item Hybrid crypto: To encrypt a large message,
    \begin{itemize}
        \item Generate random symmetric key $k$
        \item Encrypt $k$ with RSA
        \item Encrypt message with $k$
    \end{itemize}
        Sometimes share the symmetric key using RSA and use that to generate
        further keys to avoid using public-key crypto more than necessary
    \item Hybrid digital signatures: RSAsign(Hash(message))
    \item Claimed identities\\
        Suppose we get a message from ``Alice'' with a digital signature $m^d
        \mod N$. We can verify using $(m^d \mod N)^e \mod N$, but how can we be
        sure of Alice's public key if we don't know Alice?

        Use a digitial certificate (``cert''):
        \begin{itemize}
            \item Bob signs a message saying ``Alice's public key is (...)''
            \item This works if we know Bob and believe him to be trustworthy
                and competent.
            \item If we don't know Bob, then we need to ask Charlie if Bob is
                trustworthy and compentent.
            \item But if we don't know Charlie...
            \item Most common solution: pick universally trustworthy
                ``certificate authority'' who gives out keys
        \end{itemize}
\end{itemize}
