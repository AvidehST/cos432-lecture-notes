%!TEX root = InfoSec.tex
% Lecture 20: 24 November 2014
\sektion{20}{Big data and privacy}

When you have a big db with data about people, you want to make valid inferences about the population as a whole, but not valide inferences about individuals.

\textbf{Approaches}
\begin{itemize}
	\item release summary stats
	
	good privacy if done right because it withholds useful information
	\item anonymize the data set before release
	
	This is much harder to do right than it sounds because of two problems:
	\begin{itemize}
		\item quasi-identifiers: unique but "not identifying", e.g. birthday + zip code + searches for their own name
		\item linkage to outside data, e.g. looking  at netflix data and looking at IMDB data to re-identify people
	\end{itemize}
\end{itemize}

In analyzing big data, we want semantic privacy. 
\begin{definition}
	Given two databases $D$ and $D'$, where $D'$ is $D$ with your data removed, anything an analyst can learn from $D$, they can also learn from $D'$
\end{definition}

\begin{theorem}
	Semantic privacy implies that the result of the analysis does not depend on the contents of the dataset.
\end{theorem}

\begin{definition}
	A randomized function Q gives $\epsilon$-differential privacy (E-DP) if for all datasets $D_1, D_2$ that differ in the inclusion/exclusion of one element, and for all sets $S$, the following is true:

	$$\Pr[Q(D_1) \in S] \leq e^\epsilon \cdot \Pr[Q(D_2) \in S]$$

	As $\epsilon$ goes to 0, we get stricter about privacy.
\end{definition}

\sidenote{
	Useful fact: Post processing is safe. \\

	If $Q(.)$ is E-DP, then for all functions $g$, $g(Q(.))$ is E-DP.\\
}

\subsektion{How to achieve E-DP?}
Ans: Add random noise to ``true answer''

\begin{example}
Counting queries: \verb|``How many items in the DB have the property _____?''|\\

	\begin{itemize}
		\item Let C = correct answer
		\item Generate noise from distribution with probability density function (PDF) $f(x) = \epsilon/2 \cdot e^{-\epsilon|x|}$
		\item Return C + noise
	\end{itemize}
We can prove this approach is E-DP, where $\epsilon$ is a measure of accuracy/privacy. There's a tradeoff (as accuracy increases, privacy decreases).
\end{example}

To make results ``non weird'' (1) round off numbers to integers, (2) if a result is less than 0, set it to 0, and (3) if a result is greater than a cap $N$, set it to $N$.

\textbf{What about multiple queries and averaging?}
\begin{theorem}
If Q is E1-DP, and Q' is E2-DP, then (Q(.), Q'(.)) is (E1 + E2)-DP. (Essentially, you've added how non-private the queries are)
\end{theorem}

The implications? You can give an analyst an ``$\epsilon$ budget'' and let them decide how to use it.

\textbf{Generalizing beyond counting queries}\\
If each element of the database can add/subtract at most some number V from the results of the query, then we can generate noise from this distribution:
	$$f(x) = \epsilon V/2 * e^{-\epsilon V|x|}$$
and the result will be E-DP.

\subsektion{Problems}

\textbf{Collaborative recommendation systems:}\\
	\hspace*{1cm} ``People who bought X also bought Y''\\
	\hspace*{1cm} ``If you like this you'll also like that''

\begin{example}
Artificial example: But what if there exists a book that only Ed would buy? \\
	\hspace*{1cm} ``People who bought said book also bought Y''\\
Now you can easily extract information about what else Ed buys.
\end{example}

\begin{example}
Less artificial example: But what if there exists a rare book that you know Ed bought?\\
	\hspace*{1cm} ``People who bought said book also bought Y''\\
Collaborative recommendations are still a clue about what Ed bought.
\end{example}

\subsektion{How to fix}
\textbf{Look at internals}. Let's say a system computes a covariance matrix (basically a correlation between all pairs of items) to generate these recommendations. The matrix could be computed in a DP fashion, adding noise.

\textbf{Other useful tricks}\\
Machine learning + DP queries: A machine learning algorithm exchanges a bunch of DP queries and results, and it'll synthesize a new data set.
