% Lecture 2: 19 September 2012
\sektion{2}{Randomness}
Best way to get a value that is unknown to an adversary is to choose a random
value, but it's hard to get this in practice.\\

From last time, got that PRF works as MAC.

\sidenote{
    {\bf What is a PRF?}\\

    Two views:
    \begin{enumerate}
        \item family of functions $f_k(x)$
        \item function $f(k,x)$. This is the view we'll be using for this class.
    \end{enumerate}
}

{\bf True randomness}:
\begin{itemize}
    \item outcome of some inherently random process
    \item assume it ``exists'' but it's scarce and hard to get
\end{itemize}

{\bf Pseudorandom generator (PRG)}:
\begin{itemize}
    \item takes a small ``seed'' that's truly random
    \item generates a long sequence of ``good enough'' values
    \item maintains ``hidden state''
\end{itemize}

\begin{definition}
PRG is {\bf secure} if indistinguishable from a truly random
value/string.\\ This is based on the game versus Mallory (can Mallory
tell real randomness from PRG? Similar to PRF game from lecture 1),
where secure means that Mallory wins 50\% ($+\epsilon$) assuming
Mallory has limited resources.
\end{definition}

Another desireable property is Forward Secrecy:\\
If Mallory compromises the hidden state of the generator at time $t$, Mallory
can't backtrack to reconstruct past outputs of the generator.\\

Most PRGs are made up of an \textbf{init} function to initialize state
$S$ and an \textbf{advance} function to step to a new state.

\begin{example}{A PRG that is \underline{not} FS but is secure:}
    \begin{itemize}
    \item Let $f$ be a PRF
    \item init: $(seed, 0)$
    \item advance: $(seed, k) \rightarrow (seed, k + 1)$
    \item output: $f(seed, k)$
    \end{itemize}
\end{example}

\begin{example}{A PRG that is FS and secure:}
    \begin{itemize}
    \item Let $f$ be a PRF
    \item init: $seed$
    \item advance: $S \rightarrow f(S, 0)$
    \item output: $f(S, 1)$
    \end{itemize}
\end{example}

\subsektion{PRG as a system service}
Hard parts: getting seed, recovering from compromise\\

Getting a good seed: want true randomness
\begin{itemize}
    \item special circuit
    \item ambient audio/video: lava lamps! (lavarand)
\end{itemize}
problems: not \emph{truly} random (correlations)\\

Alternate view: \textbf{collect} data unpredictable to adversary
\begin{itemize}
    \item exact history of key presses
    \item exact path of mouse
    \item exact history of packet traffic
    \item periodic screenshot
    \item internal temperature
    \item ambient audio
\end{itemize}


Then: process to \textbf{extract}, or distill down to ``pure''
randomness - feed it all into a PRF. If there's enough randomness in
input, output will be ``pure random''.  Can, for example, do
SHA256(all the data).\\

 Use this to:
\begin{itemize}
    \item seed the system PRG
    \item renew the state (mix fresh randomness in with hidden state) using PRF,
        to re-establish secrecy of hidden state -- do as a precaution\\

        NOTE: Mistake to add a single bit at a time since Mallory can keep
        up with 2 possibilities at a time, but if we wait until have a
        lot, say 256 bits of randomness, then Mallory can't keep up ($2^{256}$
        possibilities), even knowing the algorithm/system since that's usually
        standard
\end{itemize}

Hard to estimate actual amount of entropy in pool, so wait for too
much randomness before mixing to remain conservative.

{\bf Linux}:
\begin{itemize}
    \item {\tt /dev/random} gives pure random bits, but have to wait
    \item {\tt /dev/urandom} is output of PRG, renewed via ``pure'' randomness
\end{itemize}

The boot problem: At startup,
\begin{itemize}
    \item least access to randomness (system is clean)
    \item highest demand for randomness (programs want keys)
\end{itemize}

Solutions (with their problems):
\begin{itemize}
    \item save some randomness only accessible at boot:\\
        hard to tell that this hasn't been observed, or used on last boot
    \item connect to someone across network to give pseudorandomness:\\
        want secure connection but don't yet have key (okay if have just enough
        for that key, or semi-predictable and hope Mallory doesn't guess)
\end{itemize}

\subsektion{Encrypting data for confidentiality}
Now may have a (passive) eavesdropper Eve:\\
\makebox[5cm]{\ovalbox{Alice} $\rightarrow$ \ovalbox{Bob}}\\
\makebox[5cm]{$\downarrow$}\\
\makebox[5cm]{\ovalbox{Eve}}\\

Message processing:\\
\makebox[1.5cm]{$\xrightarrow{\text{plaintext}}$}
    \framebox[2.5cm]{$E$ (encrypts)}
    \makebox[1.5cm]{$\xrightarrow{\text{ciphertext}}$}
    \framebox[2.5cm]{$D$ (decrypts)}
    \makebox[1.5cm]{$\xrightarrow{\text{plaintext}}$}\\
\makebox[1.5cm]{}\makebox[2.5cm]{$\uparrow$}
    \makebox[1.5cm]{}\makebox[2.5cm]{$\uparrow$}\\
\makebox[1.5cm]{}\makebox[2.5cm]{key $k$}
    \makebox[1.5cm]{}\makebox[2.5cm]{key $k$}\\

Goal: ciphertext does not convey anything about plaintext

\sidenote{
    {\bf ``encryption game'' against Mallory:}

    \hspace*{0.5 cm} Mallory chooses two pieces of plaintext\\
    \hspace*{0.5cm} We flip a coin and encrypt one of them\\
    \hspace*{0.5cm} Mallory guesses which was encrypted: wins if right

    We say that the encrpytion method is secure if Mallory can't do
    better than random guessing (50/50) + $\epsilon$.  This is known as
    \textbf{semantic security}.\\

    Note: if we were being more rigorous in our definitions, we would use a
    stronger definition of security for encryption here so that it's easier to
    combine later with integrity. However, the methods we are learning are
    secure by any of the definitions.
}

First approach: one-time pad (known to be semantically secure)
\begin{enumerate}
    \item Alice and Bob jointly generate a long random string $k$ (``the pad'')
    \item $E(k, x) = k \xor x$
    \item $D(k, y) = k \xor y = k \xor (k \xor x) = (k \xor k) \xor x = x$
\end{enumerate}
Problems:
\begin{enumerate}
    \item can't reuse key:\\
        $(k \xor a) \xor (k \xor b) = a \xor b$\\
        worst case, Eve knows one message, but even knowing that the messages
        are say English text can give Eve information from character
        distributions
    \item need really long key -- needs to be as long as sum of message lengths
\end{enumerate}
Idea: use a PRG to ``stretch'' a small key (called a ``stream cipher'')
\begin{itemize}
    \item Start with fixed-size random $k$, add a ``nonce'': unique,
      but not secret.  Use (k || nonce) to seed a PRG.
    \item Alice and Bob run identical PRGs in parallel with same key
    \item xor messages with PRG's output
    \item Do not re-use (key, nonce) pair
\end{itemize}

\subsektion{Confidentiality and integrity}
Few approaches.
\begin{enumerate}
  \item Use E(x $||$ M(x)) (done by SSL/TLS).
  \item Use E(x) $||$ M(E(x)) (done by IPSec).  This is the winner (because math).
  \item Use E(x) $||$ M(x) (done by IPSec).
\end{enumerate}

\begin{theorem}
If E is a semantically secure cipher, and M is a secure MAC, then \#2 is secure.
\end{theorem}

Encrypt plaintext, then append MAC: Bob first integrity checks, then decrypts.
Note that we need to use separate keys for confidentiality and integrity, and a
separate set of two keys for reverse channel (Bob to Alice).\\

If we have only one shared key, we seed the PRG with the shared key and then use
four  values it produces for the message sending.
